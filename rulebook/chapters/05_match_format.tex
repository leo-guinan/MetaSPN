% Chapter 5: Match Format
\chapter{Match Format}
\label{ch:match_format}

\section{Structure}

Each match lasts \textbf{3 rounds}, each representing a different task or domain.

\begin{itemize}[leftmargin=*]
  \item \textbf{Round length:} 10–15 minutes
  \item \textbf{Total match time:} Approximately 45 minutes (including timeouts and transitions)
\end{itemize}

\subsection{Timeouts}

Timeout allocation varies by class:

\begin{table}[h]
\centering
\begin{tabular}{@{}ll@{}}
\toprule
\textbf{Class} & \textbf{Timeout Allocation} \\ \midrule
Class A & 3 $\times$ 60 seconds \\
Class B & 2 $\times$ 60 seconds \\
Class C & 1 $\times$ 90 seconds + 1 reflective timeout \\ \bottomrule
\end{tabular}
\caption{Timeout Allocation by Class}
\label{tab:timeouts}
\end{table}

\textbf{Timeout rules:}
\begin{itemize}[leftmargin=*]
  \item May be called at any point during a round
  \item All communication during timeouts is publicly visible
  \item Unused timeouts do \textit{not} carry over between rounds
  \item Reflective timeouts (Class C only) may be used post-round for meta-analysis
\end{itemize}

\section{The Three Rounds}

\subsection{Round 1: Exploration}

\textbf{Focus:} Open-ended reasoning and hypothesis generation

\textbf{Typical challenges:}
\begin{itemize}[leftmargin=*]
  \item Problem framing and scope definition
  \item Generating multiple solution approaches
  \item Identifying key constraints and trade-offs
  \item Forming testable hypotheses
\end{itemize}

\textbf{Scoring emphasis:} Uniqueness (U), Learning (L), and Density (D)

\subsection{Round 2: Execution}

\textbf{Focus:} Constrained problem-solving or design

\textbf{Typical challenges:}
\begin{itemize}[leftmargin=*]
  \item Implementing a chosen strategy
  \item Optimizing under resource constraints
  \item Real-time tactical decision-making
  \item Demonstrating solution robustness
\end{itemize}

\textbf{Scoring emphasis:} Outcome Validity (O), Quality (Q), and Constraint Compliance (X)

\subsection{Round 3: Reflection}

\textbf{Focus:} Justification and meta-analysis

\textbf{Typical challenges:}
\begin{itemize}[leftmargin=*]
  \item Explaining reasoning process and decisions
  \item Analyzing strengths and weaknesses of approach
  \item Identifying alternative paths and learning points
  \item Generalizing insights to broader contexts
\end{itemize}

\textbf{Scoring emphasis:} Cohesion (C), Trust (T), and Quality (Q)

\section{Match Flow}

\begin{enumerate}
  \item \textbf{Pre-Match Briefing} (5 minutes)
  \begin{itemize}
    \item Coach and player review challenge overview
    \item Strategy formation and goal-setting
  \end{itemize}
  
  \item \textbf{Round 1: Exploration} (10–15 minutes + timeouts)
  
  \item \textbf{Transition} (2 minutes)
  \begin{itemize}
    \item Brief rest and preparation for next round
  \end{itemize}
  
  \item \textbf{Round 2: Execution} (10–15 minutes + timeouts)
  
  \item \textbf{Transition} (2 minutes)
  
  \item \textbf{Round 3: Reflection} (10–15 minutes + timeouts)
  
  \item \textbf{Post-Match Analysis} (5 minutes)
  \begin{itemize}
    \item Coach provides developmental feedback
    \item Scores finalized and published
  \end{itemize}
\end{enumerate}

\section{Match Recording \& Visibility}

All matches are:
\begin{itemize}[leftmargin=*]
  \item Recorded in full (audio, video, reasoning traces)
  \item Logged to the MetaSPN Ledger with cryptographic verification
  \item Broadcast with live Reason Map Overlay (see Chapter \ref{ch:spectator})
  \item Available for post-match review and analysis
\end{itemize}

\section{Challenges \& Appeals}

Participants may challenge referee decisions or scoring assessments:

\begin{itemize}[leftmargin=*]
  \item \textbf{In-match challenges:} Must be raised immediately; consumes one timeout
  \item \textbf{Post-match appeals:} Must be filed within 24 hours; reviewed by MetaSPN Council
  \item \textbf{Successful challenge bonus:} +0.03 to IdeaRank score for overturned calls
\end{itemize}

